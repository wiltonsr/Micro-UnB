\documentclass[conference]{IEEEtran}
\IEEEoverridecommandlockouts
% The preceding line is only needed to identify funding in the first footnote. If that is unneeded, please comment it out.
\usepackage{cite}
\usepackage[utf8]{inputenc}
\usepackage{amsmath,amssymb,amsfonts}
\usepackage{algorithmic}
\usepackage{graphicx}
\usepackage{textcomp}
\def\BibTeX{{\rm B\kern-.05em{\sc i\kern-.025em b}\kern-.08em
T\kern-.1667em\lower.7ex\hbox{E}\kern-.125emX}}
\begin{document}

\title{Sistema de Sinalização para Ciclistas\\
  % {\footnotesize \textsuperscript{*}Note: Sub-titles are not captured in Xplore and should not be used}
}

\author{\IEEEauthorblockN{1\textsuperscript{st} Karine Valença}
  \IEEEauthorblockA{\textit{Engenharia de Software} \\
    \textit{Universidade de Brasília, FGA}\\
    Gama, Brasil \\
  valenca.karine@gmail.com}
  \and
  \IEEEauthorblockN{2\textsuperscript{nd} Wilton Rodrigues}
  \IEEEauthorblockA{\textit{Engenharia de Software} \\
    \textit{Universidade de Brasília, FGA}\\
    Gama, Brasil \\
  wiltonsr94@gmail.com}
}

\maketitle

\begin{abstract}
  This document is a model and instructions for \LaTeX.
  This and the IEEEtran.cls file define the components of your paper [title, text, heads, etc.]. *CRITICAL: Do Not Use Symbols, Special Characters, Footnotes,
  or Math in Paper Title or Abstract.
\end{abstract}

\begin{IEEEkeywords}
  component, formatting, style, styling, insert
\end{IEEEkeywords}

\section{Introdução}

\subsection{Justificativa}
Este projeto tem como objetivo a criação de um sistema de sinalização com o objetivode aumentar a segurança dos ciclistas. De acordo com Departamente Nacional de
Infraestrutura de Transportes (DNIT) \cite{b1} só no ano de 2011 foram 1.698 casos
de acidentes envolvendo ciclistas. Sendo que 246, equivalente a 14.5\%, acabaram 
na morte.

\subsection{Objetivos}
O objetivo do projeto é de desenvolver um sistema de sinalização, utilizando o
MSP430, a fim de aumentar a visibilidade dos ciclistas durante seu trajeto para
aumentar a segurança e confiança dos utilizadores deste meio de transporte.

\subsection{Requisitos}

\subsection{Benefícios}
O sistema proporciona um equipamento de sinalização que ajuda os demais condutores
a ter uma melhor visão dos ciclistas. Baseado nisto o principal benefício do sistema
é a diminuição de ocorrências de acidentes envolvendo ciclistas.

\subsection{Revisão Bibliográfica}

\begin{thebibliography}{00}
  \bibitem{b1} DNIT, ``NÚMERO DE VITIMADOS ENVOLVIDOS POR TIPO DE USUÁRIO'', 2011.
\end{thebibliography}

\end{document}
