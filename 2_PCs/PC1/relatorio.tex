\documentclass[conference]{IEEEtran}
\IEEEoverridecommandlockouts
% The preceding line is only needed to identify funding in the first footnote. If that is unneeded, please comment it out.
\usepackage{cite}
\usepackage[utf8]{inputenc}
\usepackage{amsmath,amssymb,amsfonts}
\usepackage{algorithmic}
\usepackage{graphicx}
\usepackage{textcomp}
\def\BibTeX{{\rm B\kern-.05em{\sc i\kern-.025em b}\kern-.08em
T\kern-.1667em\lower.7ex\hbox{E}\kern-.125emX}}
\begin{document}

\title{Sistema de Sinalização para Ciclistas\\
  % {\footnotesize \textsuperscript{*}Note: Sub-titles are not captured in Xplore and should not be used}
}

\author{\IEEEauthorblockN{1\textsuperscript{st} Karine Valença}
  \IEEEauthorblockA{\textit{Engenharia de Software} \\
    \textit{Universidade de Brasília, FGA}\\
    Gama, Brasil \\
  valenca.karine@gmail.com}
  \and
  \IEEEauthorblockN{2\textsuperscript{nd} Wilton Rodrigues}
  \IEEEauthorblockA{\textit{Engenharia de Software} \\
    \textit{Universidade de Brasília, FGA}\\
    Gama, Brasil \\
  wiltonsr94@gmail.com}
}

\maketitle

\begin{abstract}
  This document is a model and instructions for \LaTeX.
  This and the IEEEtran.cls file define the components of your paper [title, text, heads, etc.]. *CRITICAL: Do Not Use Symbols, Special Characters, Footnotes,
  or Math in Paper Title or Abstract.
\end{abstract}

\begin{IEEEkeywords}
  component, formatting, style, styling, insert
\end{IEEEkeywords}

\section{Introdução}

\subsection{Revisão Bibliográfica}
Notícias sobre acidentes envolvendo bicicletas são comuns no Brasil. Recentemente, em São Paulo, um ciclista morreu logo após ser atropelado e arrastado \cite{b1}. Dados de 2014, mostram que 1.357 ciclistas morreram vítimas de acidentes de trânsito no Brasil, além disso, em 2016, ocorreram 11.741 internações de ciclistas vítimas de acidentes \cite{b2}. De acordo com Departamente Nacional de Infraestrutura de Transportes (DNIT) \cite{b3} só no ano de 2011 foram 1.698 casos de acidentes envolvendo ciclistas. Sendo que 246, equivalente a 14.5\%, acabaram na morte.

O site hg.org apresenta uma lista de dicas para evitar acidentes ao utilizar bicicleta. O site sugere aos ciclistas que eles se façam visíveis aos demais usuários das vias, e que utilizem sinais de mão para mostrar intenção de parar ou de mudar de faixa \cite{b4}. 

Existe uma série de sinais que podem ser utilizados pelos ciclistas para indicar suas intenções. O site mapmyrun \cite{b5}, apresenta um lista com 10 sinais que podem ser utilizados a fim de evitar acidentes. Pode-se notar que, de fato, os sinais auxiliam a diminuir os acidentes de trânsito envolvendo ciclistas. Porém, alguns desses sinais não são tão intuitivos e podem não fazer sentido para os motoristas. Além disso, a grande quantidade de sinais pode gerar confusão até mesmo aos ciclistas. 


\subsection{Justificativa}
Pode-se notar que a visibilidade e sinalização por parte dos ciclistas é crucial para sua segurança no trânsito. Diante disso, este projeto tem como objetivo a criação de um sistema de sinalização eletrônico visando aumentar a segurança dos ciclistas. Espera-se que os usuários do sistema de sinalização eletrônico sofram menos acidentes causados por falta de visibilidade.

\subsection{Objetivos}
O objetivo do projeto é de desenvolver um sistema de sinalização, utilizando o
MSP430, a fim de aumentar a visibilidade dos ciclistas durante seu trajeto para
aumentar a segurança e confiança dos utilizadores deste meio de transporte.

\subsection{Requisitos}
O sistema deve atender aos requisitos:
\begin{itemize}
\item Indicar sinal luminoso intermitente que fica ativo sempre que não houver outro sinal
\item Indicar seta para a direita ou para a esquerda após clique do botão correspondente
\item Indicar sobre parada quando o ciclista iniciar a freagem
\item Indicar sobre perigos na pista quando o ciclista apertar o botão adequado
\end{itemize}

O sistema não atende aos requisitos:
\begin{itemize}
\item Funcionar em dias chuvosos
\end{itemize}

\subsection{Benefícios}
O sistema proporciona um equipamento de sinalização que ajuda os demais condutores
a ter uma melhor visão dos ciclistas. Baseado nisto o principal benefício do sistema
é a diminuição de ocorrências de acidentes envolvendo ciclistas.


\begin{thebibliography}{00}
  \bibitem{b1} G1, ``Ciclista morre após ser atropelado e arrastado em SP''. Disponível em: http://g1.globo.com/sao-paulo/noticia/ciclista-morre-apos-ser-atropelado-e-arrastado-em-sp.ghtml.
  \bibitem{b2} G1, ``Brasil tem, em média, 32 ciclistas internados por dia devido a acidentes''. Disponível em: http://g1.globo.com/bom-dia-brasil/noticia/2017/03/brasil-tem-em-media-32-ciclistas-internados-por-dia-devido-acidentes.html.
  \bibitem{b3} DNIT, ``NÚMERO DE VITIMADOS ENVOLVIDOS POR TIPO DE USUÁRIO'', 2011.
  \bibitem{b4} Mesriani Law Group, ``Safety Tips to Avoid Bicycle Accidents''. Disponível em: https://www.hg.org/article.asp?id=7752.
  \bibitem{b5} Marc Lindsay, ``10 Cycling Hand Signals You Need to Know''. Disponível em: http://blog.mapmyrun.com/10-cycling-hand-signals-need-know/.


\end{thebibliography}

\end{document}
